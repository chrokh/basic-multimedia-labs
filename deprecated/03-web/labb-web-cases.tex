\documentclass{article}
\usepackage{amssymb}
\usepackage[T1]{fontenc}
\usepackage[utf8]{inputenc}
\usepackage{xcolor}
\usepackage{verbatim}
\usepackage{hyperref}
\usepackage{listings}
\hypersetup{%
  colorlinks=true,% hyperlinks will be black
  linkbordercolor=red,% hyperlink borders will be red
  pdfborderstyle={/S/U/W 1}% border style will be underline of width 1pt
}

\begin{document}

  \title{ WEB -- Mini-case }
  \author{ Grundläggande Multimedia | Uppsala Universitet }
  \date{}
  \maketitle

  \paragraph{}
  Detta dokument innehåller ett par mindre case som syftar till att skapa mer eller mindre kompletta webbsidor. Prioritera gärna att göra ett case ordentligt istället för att försöka stressa igenom så många som möjligt.


  \newpage
  \section{ En webb-historia }
    \paragraph{}
    Kanske har du sett att det fanns (och kanske fortfarande finns?) fysiska böcker där man kunde påverka handlingen genom att gå till olika sidor vid olika tillfällen. "Om du vill dräpa draken, gå till sida 145. Om du vill bjuda honom på kaffe, gå till 212". O.s.v. Tanken med denna uppgift är att du ska skapa just en sådan historia, där varje sida är en HTML-sida som innehåller en text, en bild och ett antal länkar. Där länkarna tar dig till andra sidor där historien fortsätter.
    \paragraph{}
    Det kan vara en fördel att göra en snabb \href{https://www.google.se/search?q=sitemap&um=1&ie=UTF-8&hl=sv&tbm=isch&source=og&sa=N&tab=wi&ei=-QVEUqT6LdD74QT69YHYCg}{sitemap} så att du kan överskåda vilka sidor som tar dig till vilka.

    \subsection*{Krav}
      \begin{enumerate}
        \item Historien ska bestå av minst 8 sidor
        \item Majoriteten av sidorna bör ha mer än ett val (länkar)
      \end{enumerate}



  \newpage
  \section{ Launching page }
    \paragraph{}
    Denna uppgift går ut på att skapa en "lanseringssida" för någonting. Vad detta någonting är, väljer du själv. Må det vara en artist, ett band, en produkt, ett spel, en webbsida, ett busskort, en reklamkampanj, en film, en tv-serie, en bok, en teater-show, eller egentligen precis vad du vill.

    \paragraph{}
    Eftersom vad vi nu bygger är en s.k. "onepage" kan \href{http://onepagelove.com/}{denna sida} nog komma till stor nytta vad gäller inspiration.

    \paragraph{}
    Försök få sidan att se så "professionell" ut som möjligt. Skulle ditt intresse väckas av den sida du skapat?




  \newpage
  \section{ Ett första steg i ett schack-spel }
    \paragraph{}
    Denna uppgift går ut på att försöka göra ett schackbräde med hjälp av HTML och CSS. Utan att använda \texttt{<table>}:s. Tanken är alltså att vi skulle kunna göra om det till ett riktigt fungerande schackspel senare, men nu börjar vi med att endast försöka få till designen av själva brädet.

    \subsection*{Tips}
      Det finns många olika sätt att lösa denna uppgift på. Nedan följer ett par tips på olika tillvägagångssätt.
      \begin{itemize}
        \item Genom att använda \texttt{<table>}:s
        \item Vi kan använda css-stilen \texttt{position:absolute;} för att lägga \texttt{<div>}:ar "ovanpå" varandra.
        \item Med hjälp av \texttt{<div>}:ar i \texttt{<div>}:ar skulle vi kunna ge "skenet" av ett rutnät.
        \item Genom att "generera" \texttt{<div>}:ar med hjälp av JavaScript.
      \end{itemize}

    \paragraph{}
      Lös denna uppgift genom \emph{minst två} metoder (men prova gärna fler!), varav den ena är genom javascript.




  \newpage
  \section{ Din portfolio }
    \paragraph{}
    Oavsett om du vill profilera dig själv som grafisk designer, illustratör, interaktionsdesigner, webbdesigner, musiker, reklambyrå, kommunikationsbyrå, spelutvecklingsfirma, programmerare, filmmakare, animatör eller någonting helt annat. Det här mini-caset går ut på att skapa en online-profil för dig själv. Alltså en portfolio där du kan showcase:a ditt arbete.

    \paragraph{}
    Portfolios innehåller ofta mycket bilder och/eller videos. Om du försöker profilera dig för någonting du ännu inte gjort, passa på att "låna" andras snygga bilder (som passar din profil) och använd dem som "placeholders" på din portfolio!

    \paragraph{}
    Tänk på att du nu försöker "sälja in" dig själv som någonting. Med andra ord. Designa professionellt! Skulle du vilja anställa digsjälv utifrån din portfolio?

    \paragraph{}
    Det finns massor av \href{http://www.portfoliobox.net/examples}{inspiration} på internet så ta dig en titt.




  \newpage
  \section{ Precis vad som helst! }
    \paragraph{}
    Det finns egentligen ingen anledning att jag ska begränsa er vilja att skapa. Har du någonting som du skulle vilja försöka förverkliga i webbside-väg, så kör på!

\end{document}