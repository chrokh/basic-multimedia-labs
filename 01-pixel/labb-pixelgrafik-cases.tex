\documentclass{article}
\usepackage{amssymb}
\usepackage[T1]{fontenc}
\usepackage[utf8]{inputenc}
\usepackage{xcolor}
\usepackage{hyperref}
\hypersetup{%
  colorlinks=true,% hyperlinks will be black
  linkbordercolor=red,% hyperlink borders will be red
  pdfborderstyle={/S/U/W 1}% border style will be underline of width 1pt
}

\begin{document}

  \title{ PIXELGRAFIK -- Mini-case }
  \author{ Grundläggande Multimedia | Uppsala Universitet }
  \date{}
  \maketitle

  \paragraph{}
  Detta dokument innehåller ett par olika case. Du behöver såklart inte färdigställa alla, så fokusera på de du finner viktiga/utmanande och försök göra de så bra du kan! De flesta case är skrivna som om du hade ett företag som kund, men vill du, är du fri att bestämma att din kund är en artist, en konstnär eller ett evenemang o.s.v.


  \newpage
  \section{ Branding \& grafisk identitet/profil }
    \paragraph{}
    Detta uppdrag handlar om att ta fram en grafisk identitet/profil för ett företag. Välj antingen ett existerande eller hitta på ett nytt. Försök gärna att tillföra någonting nytt till deras profil.

    \subsection*{ Uppgiften }
      \begin{enumerate}
        \item Välj ett företag
        \item Designa om: \emph{logotypen} om du inte vill behålla den existerande, företagets \emph{färgpalett}, samt välj vilka \emph{typsnitt} som ska ingå i identiteten.
        \item Välj ett par olika prylar, såsom visitkort, kuvert, datorer, appar, webbsidor, pennor, papper, böcker, telefonskal, montrar etc. och använd logotypen, färgpaletten och typsnittet för ingiva känslan av att alla prylarna ingår i ett och samma ekosystem.
      \end{enumerate}

      \paragraph{Resurser}
        \begin{itemize}
          \item
            Här kommer
            ett par
            \href{http://graphicburger.com/branding-identity-mock-up-vol-4/}{editerbara mallar}
            du skulle kunna
            \href{http://graphicburger.com/branding-identity-mock-up-vol-2/}{utgå ifrån}
            om du vill,
            och annars bara
            \href{http://graphicburger.com/identity-branding-mock-up-vol-3/}{använda som exempel}.
        \end{itemize}





  \newpage
  \section{Reklamaffisch}
    \paragraph{}
    Uppdraget består av att ta fram en reklamaffisch som säljer en produkt eller en tjänst.

    \subsection*{ Uppgift 1 -- Fysisk affisch}
      Reklamaffischen bör innehålla minst en \texttt{bild} och en \texttt{text}, eventuellt en \texttt{slogan}. Denna uppgift handlar alltså inte bara om de tekniska aspekterna utan även komposition, färgsättning och typsättning. Är affischen du tar fram så bra att den skulle kunna användas på riktigt? PS. Du kan ignorera tryck-aspekterna såsom att konvertera till CMYK och färgprofiler.

    \subsection*{ Uppgift 2 -- Digital banner}
     Varför stanna vid en fysisk affisch? Låt oss även nå ut med denna reklam på webben. Designa om din affisch så att den passar i ett par olika banner-format.

     På
      \href{http://en.wikipedia.org/wiki/Web_banner}{Wikipedia}
      hittar du en bild med exempel på de vanligaste banner-storlekarna.

    \subsection*{ Uppgift 3 -- Pitch}
      För att sälja in denna ide till kunden, ska vi hålla en "pitch" (tänk: 
      \href{http://www.vh1.com/celebrity/bwe/images/2008/12/Mad%20Men%20Popsicles.jpg}{serien}
      \href{http://feedingkat.typepad.com/.a/6a01127917a2cc28a40120a5a83e9f970b-800wi}{Mad}
      \href{http://farm2.static.flickr.com/1256/912519581_f241225aa2.jpg}{Men}
      ). Under denna pitch vill vi visa hur din reklamaffisch kommer att se ut på olika platser. Ditt uppdrag är alltså att på ett naturtroget och attraktivt sätt "montera" din affisch på ett antal olika ytor. Använd internet för att hitta bilder som du kan montera din affisch i.

      \subsubsection*{ Fysiska affischen }
        Kanske kan du
        \href{http://www.teamtejbrant.se/upload/images/Produkter/vadertskydd.jpg}{montera}
        din fysiska affisch i en
        \href{http://toobigtofitinhere.com/wordpress/wp-content/uploads/2010/05/IMG_1087.jpg}{busskur},
        en 
        \href{https://www.google.se/search?q=buss+uppsala&um=1&ie=UTF-8&hl=sv&tbm=isch&source=og&sa=N&tab=wi&ei=da8sUsrOFKGN7QbtnYGgDg&biw=1680&bih=929&sei=d68sUuidE4PZ4ASP_4DIBQ}{buss},
        en
        \href{http://www.laurisha.com/wp-content/uploads/2013/04/GradShow-BannerMockup.png}{skärm för en monter},
        eller en
        \href{http://www.martinholm.com/wp-content/uploads/banner_mockup.jpg}{annonsplats på en lyktstolpe}
        o.s.v.

    \subsubsection*{ Digitala affischen }
      Vad gäller din "digitala" affisch gäller det att "montera" den på några webbsidor som passar kundens målgrupp. Ta screenshots på t.ex. dn.se, tv.nu, hemnet.se eller någon annan sida som passar målgruppen och ersätt annonserna med din egen annons.

      Att få en screenshot av en webbsida att se attraktiv är inte alltid enkelt, så ta gärna inspiration ifrån webb-uppgiften. Den pratar om att
      \href{http://dribbble.com/shots/1226243-Wp-Theme-Experiment?list=popular&offset=1}{sätta bilderna}
      i en mer
      \href{http://dribbble.com/shots/1226541-New-site-up?list=popular&offset=15}{verklig kontext}.

  \newpage
  \section{Webb Wireframes \& Mockups}
    \paragraph{}
    Webbsidor antar ofta många skepnader innan de når sitt "slutgiltiga" utseende. Eftersom det generellt är mer kostsamt (i tid) att "koda om" ett gränssnitt än att "rita om" ett, så brukar vi premiera det sistnämnda.

    \paragraph{}
    Många webbsdesigners börjar med en \texttt{wireframe}, som är en mycket enkel representation av webbsidan bestående av boxar och korta förklarande texter, vars huvudsakliga syfte är att kommunicera layouten, snarare än designen. 

    Vidare arbetar många fram en \texttt{mockup} som är en grafiskt rikare representation av sidan. I många fall är mockupen hur designern vill att sidan ska se ut, pixel för pixel.

    Beronde på projekt, kan det hända att sidan ska ingå i en \texttt{pitch} till kund, och då kan det hända att vi behöver sätta sidan i en kontext för att göra bilden mer attraktiv.

    \paragraph{}
    Din uppgift är att först ta fram en wireframe, sedan överföra den till en mockup, och slutligen montera den färdiga designen i en egen bild.

    \subsection*{Uppgift 1 -- Wireframe}
      En
      \href{http://blog.radarhill.com/wp-content/uploads/2012/09/wireframe01.jpg}{wireframe}
      bör vara så
      \href{http://3.bp.blogspot.com/-VHAxXdC678w/T05IKQILbWI/AAAAAAAAAYE/_fIZt0IQEew/s1600/Wireframe_WebPeekaboo-ShopAge.png}{enkel}
      som möjligt,
      för att vi snabbt ska kunna
      \href{http://www.cesarinla.com/boomers/images/project/web_wireframe/web_wireframe02_04.jpg}{iterera}
      och förändra den
      tills vi är
      \href{http://www.wirify.com/wp/wp-content/uploads/2011/02/CNN-International-Original-vs-Wirify-wireframe1.jpg}{nöjda}.

      \begin{enumerate}
        \item Välj en sida (ex: facebook.com, sj.se, youtube.com etc.).
        \item Skriv en lista över funktionalitet, länkar och innehåll.
        \item Skapa en wireframe som beskriver den nya design du föreslår
      \end{enumerate}

      \subsubsection*{Resurser}
      \href{http://960.gs/}{960 grid system}
      är ett mycket värdefullt verktyg när man arbetar med wireframes. Photoshop har "rulers" (linjaler) som du kan få fram genom att trycka \texttt{cmd+r} och sedan "dra ut en linjal". Men att använda ett fördefinerat grid sparar ofantligt mycket tid och leder ofta till bra resultat.
      \href{http://uxmovement.com/wp-content/uploads/2010/12/textimages.png}{Här}
      kommer
      \href{https://drupal.org/files/images/OMEGA_WIREFRAME-FIRST-BODY-LAST.preview.png}{några}
      bilder
      som borde ge dig en förståelse för
      \href{http://images.sixrevisions.com/2010/07/12-03_12columngrid.jpg}{varför}
      det är så
      \href{http://netdna.webdesignerdepot.com/uploads/grid_based_layout_designs/woothemes-960gs-gutters.jpg}{värdefullt}.

    \subsection*{Uppgift 2 -- Mockup}
    Skapa en \texttt{mockup} som utgår ifrån din \texttt{wireframe}. Gör detta i ett nytt dokument så att du inte skriver över din wireframe.

    \subsection*{Uppgift 3 -- Pitch}
    För att öka chanserna att kunna sälja denna ide till kunden så ska vi höja nivån ett snäpp. För att sälja iden till kund ska vi försöka höja nivån ett snäppsä. Välj någon eller en kombination av nedanstående:
      \begin{itemize}
        \item{Lägg in din mockup i en webbläsare. }
        För att sätta sidan i sin mer naturliga kontext. Kom ihåg att ändra webbläsarens titel och adressfält. En sätt att göra sidan ännu mer levande är att ge webbläsaren en mycket lätt rotation, eller lite, lite perspektiv. Lägg även in en passande bakgrund bakom webbläsaren.
        \item{ Använd ett foto på en dator }
        Gå ett steg längre, genom att hämta ett snyggt foto på en dator med en webbläsare framme. Klipp sedan in din mockup i datorns skärm.
        \item {Belys snygga/viktiga detaljer. }
        Genom att kopiera en del av mockupen och göra den större. Som om någon hade varit framme med ett förstoringsglas.
      \end{itemize}

      \paragraph{}
      Ovan nämnda tekniker - använda rätt - kan göra underverk. Även för de svagaste av mockups!

      \paragraph{Exempel på pitchar}
      \begin{itemize}
        \item
          Att visa sin mockup i en
          \href{http://netdna.webdesignerdepot.com/uploads/2013/07/browsermockup.jpg}{webbläsare}
          mot enkel
          \href{http://www.pixeden.com/media/k2/galleries/196/002-google-chrome-browser-navigator-web-psd.jpg}{bakgrund}
          kan göra stor skillnad!
        \item
          Att
          \href{http://freebiesbug.com/wp-content/uploads/2013/03/free-psd-safari-browser.jpg}{belysa}
          några
          \href{http://dribbble.com/shots/308945-Texture-Detailing?list=searches&tag=website_details}{detaljer}
          kan vara ett bra sätt att få kunden att märka de
          \href{http://dribbble.com/shots/659616-Project-Planner-details?list=searches&tag=website_details}{mindre saker}
          vi lagt extra
          \href{http://dribbble.com/shots/311028-Donate?list=searches&tag=website_details}{kärlek}
          på.
        \item
          För att ge ännu mer liv till bilden kan vi
          \href{http://multimediabomb.com/preview/thumbs/hands-on-macbook-laptop-mockups-psd.jpg}{montera}
          mockup:en i en
          \href{http://dribbble.com/shots/1226683-Portfolio-Redesign-2k13?list=tags&tag=website}{Macbook Pro},
          en
          \href{http://dribbble.com/shots/1226951-Minimal-Browser-Version?list=popular&offset=70}{Macbook Air},
          kanske en
          \href{http://kravencastle.com/wp-content/uploads/2012/09/iMac_mari01.jpg}{iMac},
          eller
          \href{http://dribbble.com/shots/1227262-Website-Portfolio?list=tags&tag=website}{varför}
          inte en
          \href{http://dribbble.com/shots/1224075-Central-Reach?list=tags&tag=website}{ipad},
          eller helt enkelt
          \href{http://www.psdexplorer.com/wp-content/uploads/2012/09/free-psd-responsive_screen_mockup_pack.png}{flera enheter}
          samtidigt.
        \item
          Har vi mycket att säga kan vi använda oss av
          \href{http://f.cl.ly/items/0c2v1D1x0S463X0O151u/marketing-presentation-minimal-landing.png}{berätta en historia}
          utanför webbläsaren.
      \end{itemize}

      \paragraph{Resurser}
      \begin{itemize}
        \item
          \href{http://www.mohunky.com/content/public/v7/BrowserMock-upPSDs.aspx}{Bilder på webbläsare i PSD-format}
        \item
          Vektor-representationer av  
          \href{http://www.bestpsdfreebies.com/freebie/vectorized-imac/}{iMac}, 
          \href{http://www.bestpsdfreebies.com/freebie/vectorized-macbook-pro/}{Macbook Pro}, 
          \href{http://www.bestpsdfreebies.com/freebie/ipad-web-preview-mockup/}{iPad}, 
          \href{http://www.bestpsdfreebies.com/freebie/macbook-air/}{Macbook Air}
      \end{itemize}



      % Länkar för app-mocks
      % http://freebiesbug.com/psd-freebies/perspective-screens-mockup/
      % http://graphicburger.com/iphone-5-angle-view-mock-up/
      % http://www.google.se/imgres?start=135&um=1&hl=sv&biw=1680&bih=885&tbm=isch&tbnid=lxYAYFcK8OUxmM:&imgrefurl=http://blogs.missouristate.edu/computerscience/&docid=W8JH5FZh1oydyM&imgurl=http://blogs.missouristate.edu/computerscience/files/2013/08/mockup.jpg&w=550&h=445&ei=Z70sUunUKMn74QTHh4HQDw&zoom=1&ved=1t:3588,r:50,s:100,i:154&iact=rc&page=5&tbnh=169&tbnw=243&ndsp=32&tx=144&ty=46




\end{document}