\documentclass{article}
\usepackage{amssymb}
\usepackage[T1]{fontenc}
\usepackage[utf8]{inputenc}
\usepackage{xcolor}
\usepackage{hyperref}
\hypersetup{%
  colorlinks=true,% hyperlinks will be black
  linkbordercolor=red,% hyperlink borders will be red
  pdfborderstyle={/S/U/W 1}% border style will be underline of width 1pt
}

\begin{document}

  \title{ PIXELGRAFIK -- Övningsuppgifter }
  \author{ Grundläggande Multimedia | Uppsala Universitet }
  \date{}
  \maketitle


  \section{ Reflektions-effekt }
    \paragraph{}
    Dags för reklambilder! Vi ska nu försöka skapa en reflektions-effekt. En sådan som ofta används för produktbilder eller på webbsidor. Sök på \href{https://www.google.se/search?q=engraved+text+photoshop&um=1&ie=UTF-8&hl=sv&tbm=isch&source=og&sa=N&tab=wi&ei=xK4lUsD7DNSQhQep94CYBQ&biw=1280&bih=679&sei=xq4lUuefCMGJhQfghYCoAw#hl=sv&q=image+reflection+photoshop&tbm=isch&um=1}{image reflection photoshop} för att förstå vad vi pratar om.

    \paragraph{ Övning }
      \begin{enumerate}
        \item Välj en bild på en produkt, och frilägg produkten.
        \item Skapa en reflektionseffekt som gör att produkten ser attraktivare ut.
        \item Lägg till någon form av bakgrund för att ge bilden en helhetskänsla.
      \end{enumerate}



  \section{Byta ut färg}
    \paragraph{}
    Hitta ett fotografi på vad som helst. Vi ska försöka förändra en "color range" i bilden så tänk på det när du väljer bild. Om du hittar ett schackbräde skulle du kunna försöka förändra den ena "färgen" i brädet. Eller kanske en persons solglasögon. Eller bladen på en blomma.

    \paragraph{Övning}
      \begin{enumerate}
        \item Bestäm dig för en färgnyans i bilden du vill förändra. Förändra den markant, men försök få det att se så naturligt ut som möjligt. Hackiga och pixliga lösningar är inte lösningar.
        \item Börja om ifrån din orginalbild igen och försök nu att göra en färgförändring på ett avgränsat område. Om du t.ex. har en bild på en blomma och en sol, där båda går i gult. Om vi då förändrar den gula nyansen så kommer även solen att förändra färg. Försök alltså att endast förändra färgen på blomman.
      \end{enumerate} 

    \paragraph{Tips på verktyg \& inställningar}
      \begin{itemize}
        \item Image > Adjustments > Replace color
        \item Select > Color range
        \item Markeringsverktyg
        \item Håll in shift eller alt för lägga till eller ta bort ifrån en befintlig markering när du markerar.
        \item Image > Adjustments > Hue/Saturation
        \item Image > Adjustments > Brightness/Contrast
        \item Brush (med opacitet)
        \item Lager (med opacitet)
      \end{itemize}


  \section{Spara för webben}
    \paragraph{}
    Vi ska nu prova att spara i lite olika format. Hämta ett högupplöst fotografi ifrån internet och öppna det i Photoshop.

    \paragraph{Övning}
      \begin{enumerate}
        \item Använd "Spara för webben" för att spara bilden i följande olika format (.png, .jpg, .gif).
        \item Spara två kopior av varje format, en större (ex: 800px i bredd) och en mindre (ex: 150px i bredd)
        \item Vilka format ger större bilder? Vilka format ger mindre bilder? Varför?
        \item Återupprepa hela övningen, fast istället för att spara ett foto, fyll istället hela bilden med en solid färg och spara igen. Vad får vi för resultat nu? Varför?
      \end{enumerate}


  \section{Rädda ett gammalt foto!}
    \paragraph{}
    Hitta ett gammalt och skruttigt foto. En google sökning på \href{https://www.google.se/search?biw=1680&bih=882&q=dusty+old+photo&bav=on.2,or.r_qf.&bvm=bv.51495398,d.ZG4,pv.xjs.s.en_US.M4-36_38X9A.O&um=1&ie=UTF-8&hl=sv&tbm=isch&source=og&sa=N&tab=wi&ei=cKEkUuzdHYGVhQeytYC4Bw#hl=sv&q=old+photograph&tbm=isch&um=1}{"old photo"} borde hjälpa.
    
    \paragraph{Övning}
      \begin{enumerate}
        \item Försök rädda bilden ifrån vikningar, damm, skrapningar och allt annat smuts. Prova dig fram och försök få fram en bild du är mer nöjd med!
      \end{enumerate} 

    \paragraph{Tips på verktyg \& inställningar}
      \begin{itemize}
        \item Filter > Noise > Dust \& scratches
        \item Filter > Noise > Median
        \item Clone stamp
        \item Spot healing brush
        \item Sharpen eller Blur
        \item Markeringsverktyg
      \end{itemize}


  \section{Skapa bakgrundstextur}
    \paragraph{}
    Vi ska nu försöka skapa en bakgrund som påminner om trä, betong, vatten, eller liknande textur. ifrån ingenting. Börja alltså med ett tomt dokument i rimlig storlek för en bakgrund. Googla fram en bild du vill använda som referens, ex: "wooden background", "concrete background" eller dyl.

    \paragraph{Övning}
      \begin{enumerate}
        \item Börja med ett tomt dokument i rimlig bakgrunds-storlek.
        \item Börja med ett filter under "Filter > Render", och experimentera dig sedan fram i små steg för att försöka efterlikna din referensbild.
      \end{enumerate}

    \paragraph{Tips på verktyg \& inställningar}
      \begin{itemize}
        \item Brush (pensel)
        \item Opacitet
        \item Hue/Saturation, Brightness/Contrass, Levels, Curves
        \item Adjustment layers
        \item Filters (bl.a. Distort > Wave; Noise > Add Noise; Pixelate > Crystallize, Distort > Ripple)
        \item Blending modes
      \end{itemize}


  \section{ Texturer över text}
    \paragraph{}
    Vi ska nu försöka skapa en text som ser ut som att den är "inbankad" ("engraved") i ett stycke betong eller trä (\href{https://www.google.se/search?q=engraved+text+photoshop&um=1&ie=UTF-8&hl=sv&tbm=isch&source=og&sa=N&tab=wi&ei=xK4lUsD7DNSQhQep94CYBQ&biw=1280&bih=679&sei=xq4lUuefCMGJhQfghYCoAw#hl=sv&q=wooden+text+photoshop&tbm=isch&um=1}{sök på "wooden text photoshop"} för exempel). 

    \paragraph{Övning}
      \begin{enumerate}
        \item Hämta en bakgrundsbild föreställande t.ex. trä eller betong.
        \item Lägg den i ett lager längst ned. Skriv sedan en text i ett lager ovanför.
        \item Försök nu få texten att se ut som att den är graverad i materialet.
      \end{enumerate}

    \paragraph{Tips på verktyg \& inställningar}
      \begin{enumerate}
        \item Layer styles
        \item Layer opacity ("Fill")
        \item Blending modes
        \item cmd-klicka på ett lagers "minatyr-bild" för att skapa en markering som ser ut som det som finns i lagret.
      \end{enumerate}
  



  \section{Friläggning \& Montage}
    \paragraph{}
    Hitta två bilder och arbeta med friläggning. Gärna en landskapsbild och en bild med personer eller djur.

    \paragraph{Övning 1}
      \begin{enumerate}
        \item Ta bort ett objekt (person el. djur) ifrån den ena bilden utan att lämna några "spår''.
      \end{enumerate}

      \paragraph{Övning 2}
      \begin{enumerate}
        \item Flytta ett objekt (person el. djur) ifrån en bild och "montera" objektet i en annan bild (bakgrund).
        \item Försök få det att se så naturligt ut som möjilgt.
      \end{enumerate}

    \paragraph{Tips på verktyg \& inställningar}
      Utan inbördes ordning.
      \begin{itemize}
        \item Markeringsverktyg
          \begin{itemize}
            \item Quick selection
            \item Magic Wand
            \item Magnetic lasso
          \end{itemize}
        \item Markeringsförändrare \& finjustering
          \begin{itemize}
            \item Select > Modify (Markera > Ändra)
            \item Quick mask
            \item Håll shift eller alt medan du markerar för att lägga till/ta bort ifrån en befintlig markering
          \end{itemize}
        \item Fylla i där man tagit bort
          \begin{itemize}
            \item Spot healing brush
            \item Clone stamp
            \item Edit > Fill > Content Aware
          \end{itemize}
      \end{itemize}


      
  \section{ Avancerat montage }
    \paragraph{}
      Försök montera dig själv på en skylt!
    \paragraph{Övning}
      \begin{enumerate}
        \item Hitta en bild som innehåller en vägskylt, reklamskylt eller dyl.
        \item Ta en bild på dig själv med Photo Booth
        \item Försök "montera" dig själv i skylten så att det ser så verkligt ut som möjligt.
      \end{enumerate}



\end{document}