\documentclass{article}
\usepackage{amssymb}
\usepackage[T1]{fontenc}
\usepackage[utf8]{inputenc}
\usepackage{xcolor}
\usepackage{verbatim}
\usepackage{hyperref}
\usepackage{listings}
\hypersetup{%
  colorlinks=true,% hyperlinks will be black
  linkbordercolor=red,% hyperlink borders will be red
  pdfborderstyle={/S/U/W 1}% border style will be underline of width 1pt
}

\begin{document}

  \title{ Ljud \& musik -- Mini-case }
  \author{ Grundläggande Multimedia | Uppsala Universitet }
  \date{}
  \maketitle

  \paragraph{}
  Detta dokument innehåller ett par exempel på mindre case som syftar till att träna dina kunskaper i ljudframtagning och -redigering. Men dessa är som sagt bara exempel. Så om du känner dig mer manad att bara experimentera i programmet -- kör på!


  \newpage
  \section{ Radioteater }
    \paragraph{}
    Det finns många gratis ljudböcker på nätet. Leta rätt på en och dra in ett kapitel i ditt ljudprogram. Leta sedan successivt rätt på ljud och effekter som du kan använda för att skapa "atmosfär" i ljudboken.
    \paragraph{}
    Om vi t.ex. har raden "As Gregor Samsa awoke one morning from uneasy dreams he found himself transformed in his bed into a monstrous vermin" så skulle vi kanske först vilja ha ljudet av en gäspning, och sedan ett uttryck av förvåning. Kanske skulle vi även vilja ha lite atmosfär runt om kring -- vad sägs t.ex. om ett vårljud av fågelkvitter (utanför fönstret), eller det nedtonade ljudet av förbipasserande trafik.
    \paragraph{}
    Tänk på att du kan använda dig av effekter anpassa ljudet. Om vi t.ex. vill ha ljudet av trafik utanför ett fönster så kanske vi vill "dämpa" ljudet lite för att ge illusionen av att det befinner sig utanför fönstret.



  \section{ Återskapa en låt }
    \paragraph{}
    Välj en låt du kan. Försök sedan att steg för steg återskapa låten. Antingen kan du ta någonting enkelt som "Bä bä vita lamm" eller "Ja må du leva". Alternativt så kan du ta en "helt vanlig" låt och försöka återskapa den steg för steg - t.ex. börja med melodin, trummor, sen bas, o.s.v.



  \section{ Film-soundtrack }
    \paragraph{}
    Hitta en film på youtube, och börja sedan skapa ett soundtrack som du tycker passar till filmen!



  \section{ Remix }
    \paragraph{}
    Ladda ned valfri låt ifrån internet. Börja sedan experimentera med låten. Gör en remix! Kanske vill du göra en snabbare låt. Kanske kan du göra om den till dubstep? Kanske kan du göra om den till ett hiphop-beat.
    \paragraph{}
    Experimentera med att "sampla" olika delar av låten, loopa, klipp och experimentera!


  \section*{ Lycka till! }

\end{document}