\documentclass{article}
\usepackage{amssymb}
\usepackage[T1]{fontenc}
\usepackage[utf8]{inputenc}
\usepackage{xcolor}
\usepackage{color}
\usepackage{verbatim}
\usepackage{hyperref}
\usepackage{listings}
\lstdefinelanguage{CSS}{
    keywords={hover,color,background-image,margin,padding,font,weight,display,position,top,left,right,bottom,list,style,border,size,white,space,min,width, transform, transition, transition-property, transition-duration, transition-timing-function},
    sensitive=true,
    morecomment=[l]{//},
    morecomment=[s]{/*}{*/},
    morestring=[b]',
    morestring=[b]",        
    alsoletter={-},
    alsodigit={:}
}
\lstdefinelanguage{JavaScript}{}
\lstset{
  keywordstyle=\color[rgb]{0,0,1},
  commentstyle=\color[rgb]{0.133,0.545,0.133},
  stringstyle=\color[rgb]{0.627,0.126,0.941},
}

\hypersetup{%
  colorlinks=true,% hyperlinks will be black
  linkbordercolor=red,% hyperlink borders will be red
  pdfborderstyle={/S/U/W 1}% border style will be underline of width 1pt
}

\begin{document}

  \title{ HTML5 och Javascript -- Övningsuppgifter }
  \author{ Grundläggande Multimedia | Uppsala Universitet }
  \date{}
  \maketitle


  \section{ Introduktion }
    Detta dokument innehåller små övningsuppgifter du kan använda för att träna dina multimediakunskaper i HTML5 och Javascript. Se till skapa en \emph{ny mapp för varje labbuppgift} under denna labb!


  \paragraph{}
  Eftersom vi kommer att jobba med webbsidor är det viktigt att du kommer ihåg några saker.
    \begin{enumerate}
      \item Se till att alla dina .html-dokument minst innehåller det minsta du behöver för att vara korrekt enligt HTML5-standarden (se nedan för ett exempel).
            \lstset{language=HTML}
            \begin{lstlisting}
  <!DOCTYPE html>
  <html>
    <head>
      <title>Your title here</title>
    </head>
    </body>
      <!-- Your content here -->
    </body>
  </html>
            \end{lstlisting}
      \item Se till att du \href{http://htmlhunden.se/#indentering}{indenterar korrekt}!
      \item Glöm inte att du alltid kan läsa igenom \href{http://htmlhunden.se}{HTMLHunden} om du känner dig förvirrad.
    \end{enumerate}


  \section{ Uppgifter }
  Nedan följer uppgifterna. Försök att hinna med allihopa!


  \subsection{ HTML5 Audio }
    I den här övningen ska vi använda oss av audio-elementet i HTML5. Du hittar information om hur audio-elementet fungerar bl.a. hos \href{http://www.w3schools.com/html/html5_audio.asp}{W3Schools}.

    \paragraph{Uppgiften}
      \begin{enumerate}
        \item Skapa en ny html-sida
        \item Leta rätt på en ljudfil i mp3-format
        \item Placera ljudfilen i samma mapp som din html-sida
        \item Använd audio-elementet för att spela upp ljudet på din html-sida
      \end{enumerate}


  \subsection{ HTML5 Video }
    I den här övningen ska vi använda oss av video-elementet i HTML5. Du hittar information om hur video-elementet fungerar bl.a. hos \href{http://www.w3schools.com/html/html5_video.asp}{W3Schools}.

    \paragraph{Uppgiften}
      \begin{enumerate}
        \item Skapa en ny html-sida
        \item Leta rätt på en videofil i ett format som stödjs av HTML5 video.
        \item Lägg INTE in videofilen i din mapp utan kopiera istället URL:en till videon.
        \item Använd video-elementet i din html-fil och peka src-attributet till den URL du tidigare kopierade.
      \end{enumerate}



\newpage
\section{ Koppla in en Javascript-fil }
\paragraph{}
I denna uppgift kopplar vi in ett Javascript-dokument.

\subsection*{Uppvärmning}
\begin{enumerate}
	\item Skapa en \texttt{.html}-fil
	\item Skapa en \texttt{.js}-fil
	\item Lägg följande kod i \texttt{.js}-filen:
	\begin{lstlisting}
	var name = "Your name";
	alert("Hello " + name);
	\end{lstlisting}
	\item Om du nu kopplar in \texttt{.js}-filen i \texttt{.html} med hjälp av \texttt{<script>}-elementet så bör du få ett 
	
	välkomnstmeddelande när du öppnar Javascript-sidan.
	\item Experimentera gärna lite med \texttt{.js}-filen. Försök t.ex. få ditt namn att visas istället. Eller ett slumpmässigt 
	
	nummer.
\end{enumerate}

  \subsection{ Kontrollera ljuduppspelning genom JavaScript }
    \paragraph{}
    I denna uppgift ska vi försöka skapa egna "kontroller" i en HTML-sida som vi kan använda för att genom JavaScript kontrollera uppspelningen av HTML5-ljud.

    \subsubsection*{Uppgiften}
      \begin{enumerate}
        \item Skapa en ny .html-sida
        \item Leta rätt på en ljudfil och spela upp den på din sida genom HTML5 audio.
        \item Skapa tre länkar \emph{<a>} eller buttons \emph{<button>}: PLAY, JUMP FORWARD, JUMP BACKWARDS
        \item Skapa en ny .js-fil och koppla in den till ditt html-dokument.
        \item Vi ska nu genom JavaScript försöka kontrollera uppspelningen av vårt HTML5-ljud. Tanken är alltså att när användaren trycker på PLAY så börjar ljudet spelas upp, och när användaren trycker på PLAY igen så pausas ljudet. När användaren trycker på JUMP FORWARD så hoppar spelarhuvudet framåt någon/några sekunder. Och när användaren trycker JUMP BACKWARDS så hoppas spelhuvudet bakåt någon/några sekunder.
        \item Konsultera internet för att hitta \href{https://msdn.microsoft.com/en-us/library/gg589489(v=vs.85).aspx}{guider som förklarar hur man kan interagera med HTML5 audio genom JavaScript}.
      \end{enumerate}

  \subsection{ Kontrollera videouppspelning genom JavaScript }
    \paragraph{}
    Denna uppgift går ut på att göra exakt samma sak som i uppgiften där vi ska kontrollera HTML5-uppspelning med JavaScript förutom uppgiften nu är att kontrollera \href{http://msdn.microsoft.com/en-us/library/ie/hh924823(v=vs.85).aspx}{HTML5-videouppspelning med JavaScript}.



\newpage
\section{ Använda ett JavaScript-ramverk }
\paragraph{}
Denna övning går ut på att koppla in ett tredjepartsramverk för JavaScript. Det finns många olika att välja på men vi ska använda det populära 
\href{http://jquery.com/download/}{jQuery}.

\subsection*{Uppgifter}
\begin{enumerate}
	\item Hämta hem js-ramverket jQuery
	\item Skapa en html-sida och "koppla" jQuery till din html-sida
	\item Skapa en till JavaScript-fil som du kopplar in till html-sidan (det är i denna fil vi kommer använda ramverket)
	\item Använd js-ramverket till att försöka uppnå en eller flera av nedanstående funktionaliteter:
	\begin{enumerate}
		\item Lägg till en länk. När man klickar på länken ska en text visas i en \texttt{alert}-ruta.
		\item Lägg till en länk på sidan. När man klickar på länken ska bakgrundsfärgen ändras till en slumpmässig färg (ny färg 
		
		varje gång)
		\item Lägg till en bild på sidan. När man klickar på bilden ska den bli dubbelt så stor. Klickar man igen ska den återgå 
		
		till den mindre storleken.
	\end{enumerate}
\end{enumerate}

\newpage
\section{ Animering \& interaktion }
\paragraph{}
Denna övning utforskar olika sätt att animera. Skapa ett nytt html-dokument med tillhörande css- och JavaScript-filer. Skapa en 

\texttt{<div>} i html-dokumentet och skriv en text i \texttt{<div>}:en. Använd css för att ge boxen en bakgrundsfärg.

\subsection*{Uppgiften}
\begin{enumerate}
	\item Animera boxen så att den "glider in" på sidan.
	\item Lägg till en länk i boxen (förslagsvis ett kryss). När man klickar på länken ska boxen "glida ut" igen.
\end{enumerate}

\paragraph{}
Vi kan närma oss den här uppgiften ifrån två huvudsakliga håll. Antingen försöker vi lösa den med hjälp av CSS3. Eller så 

försöker vi lösa den med JavaScript (rimligen m.h.a. ett ramverk). Försök gärna båda sätten!

\paragraph{}
Tips på \texttt{jQuery}-metoder som kan användas är förslagsvis (\texttt{slideDown}, \texttt{slideUp}, \texttt{slideToggle}, 

eller \texttt{animate})





  \subsection{ Mouse Over med jQuery }
    \paragraph{}
    I denna uppgift ska vi göra exakt samma saker som i Mouse Over-uppgiften som fokuserade på CSS. Fast istället för att använda CSS ska vi nu använda javascript-biblioteket \href{http://jquery.com/download/}{jQuery}.

    \subsubsection*{Uppgiften}
      \begin{enumerate}
        \item Skapa en ny .html-sida
        \item Ladda ned \href{http://code.jquery.com/jquery-1.10.2.min.js}{jQuery} och spara filen som \emph{jquery.js}, i den mapp som innehåller din nya .html-sida.
        \item Koppla in \emph{jquery.js} till ditt html-dokument som precis vilken annan javascript-fil som helst.
        \item Skapa en ny .js-fil, döp den till \emph{main.js} och öppna den. Det är nu här vi ska skriva vår javascript-kod som använder sig av jQuery-biblioteket.
        \item Skriv följande kod i \emph{main.js}:
          \lstset{language=JavaScript}
          \begin{lstlisting}
  $(function(){
    alert("Hello from jQuery!");
    // Put all your code here
  });
          \end{lstlisting}
        \item Om du nu öppnar din .html-sida i en webbläsare bör du få ett hälsningsmeddelande ifrån webbläsaren.
        \item Ta bort \emph{alert}-raden.
        \item Skriv istället följande kod i \emph{main.js} där texten "Put all your code here står".
          \begin{lstlisting}
  $('a').hover(function(){
    (this).css('color', 'red');
  });
          \end{lstlisting}
        \item Ovan kod lägger alltså, genom jQuery, till en event-lyssnare på alla \emph{<a>}-taggar. Således kommer den anonyma funktionen köras när användaren hovrar över en a-tagg.Funktionen kommer först att välja det element som hovrats, och sedan säga åt det att tilldela den stilregeln som säger att dess färg ska vara röd.
        \item Låt oss prova en regel till. Klistra även in följande kod i \emph{main.js}:
        \begin{lstlisting}
  $('img').hover(function(){
    $(this).css('opacity', 0.5);
  });
        \end{lstlisting}
        \item Fortsätt nu att skriva mer javascript tills du uppnåt all funktionalitet som vi i förra uppgiften uppnådde genom css.
      \end{enumerate}








 

\end{document}