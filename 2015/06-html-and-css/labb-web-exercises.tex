\documentclass{article}
\usepackage{amssymb}
\usepackage[T1]{fontenc}
\usepackage[utf8]{inputenc}
\usepackage{xcolor}
\usepackage{verbatim}
\usepackage{hyperref}
\usepackage{listings}
\hypersetup{%
  colorlinks=true,% hyperlinks will be black
  linkbordercolor=red,% hyperlink borders will be red
  pdfborderstyle={/S/U/W 1}% border style will be underline of width 1pt
}

\begin{document}

  \title{ HTML och CSS -- Övningsuppgifter }
  \author{ Grundläggande Multimedia | Uppsala Universitet }
  \date{}
  \maketitle

  \paragraph{}
  Detta dokument innehåller små övningsuppgifter du kan använda för att träna dina HTML-, och CSS-kunskaper.


  \newpage
  \section{ Ett enkelt dokument }
    \paragraph{}
    Skapa en enkel HTML-sida som innehåller allt i nedanstående lista. I denna övning håller vi oss till HTML (struktur) och ignorerar CSS (stil).

    \subsection*{Sidan ska innehålla}
      \begin{enumerate}
        \item En rubrik
        \item En paragraf
        \item En numrerad lista
        \item En onumrerad lista
        \item En nästlad lista (lista i en lista)
        \item En \href{https://www.google.se/search?q=kitten+meme&tbm=isch}{bild}
        \item En länk till en annan webbsida på nätet
      \end{enumerate}


  \newpage
  \section{ Navigation }
    \paragraph{}
    I denna uppgift jobbar vi länkar. Tanken är att du ska skapa tre separata sidor -- låt oss kalla dem \texttt{A}, \texttt{B}, och \texttt{C}. Varje sida ska innehålla länkar till någon eller några av de andra sidorna. För länkarna gäller följande:

    \subsection*{Uppgiften}
      \begin{enumerate}
        \item \texttt{A} ska länka till \texttt{B} och \texttt{C}
        \item \texttt{B} ska länka till \texttt{C} och \texttt{A}
        \item \texttt{C} ska länka till \texttt{A}
      \end{enumerate}

    \paragraph{}
    Se till att du skriver någonting på varje sida för att enklare hålla koll på vilken sida som är vilken.




  \newpage
  \section{ Koppla in ett stylesheet }
    \paragraph{}
    I denna uppgift kopplar vi in ett CSS-dokument.

    \subsection*{Uppgiften}
      \begin{enumerate}
        \item Skapa en \texttt{.html}-fil
        \item Skapa en \texttt{.css}-fil
        \item Lägg följande kod i \texttt{.css}-filen:
          \begin{lstlisting}
  body{
    background: red;
  }
          \end{lstlisting}
        \item Om du nu kopplar in \texttt{.css}-filen i \texttt{.html} med hjälp av \texttt{<link>}-elementet så bör sidan vara röd när du öppnar den i en webbläsare.
        \item Lägg in lite content i din \texttt{.html}-fil, såsom text, länkar, rubriker etc. och experimentera lite med CSS genom att skriva olika regler för olika element ditt \texttt{.html}-dokument.
      \end{enumerate}



  \newpage
  \section{ Blogginlägg }
    \paragraph{}
    I denna uppgift kopplar vi in ett CSS-dokument och arbetar med sidans visuella stil. Utgå ifrån resultatet ifrån ovan uppgift eller skapa en helt ny HTML-sida. Tanken är vi ska skapa en sida som ser ut som ett blogginlägg. Försök få sidan att se så professionell ut som möjligt. Alltså inget Comic Sans :)

    \subsection*{Sidan ska innehålla}
      \begin{enumerate}
        \item Bloggpostens titel
        \item Meta-information om bloggposten, där följande nyckelkord ska särskiljas med hjälp av en annan färg ("färgkodas").
          \begin{enumerate}
            \item Skribent
            \item Datum
            \item Kategori
          \end{enumerate}
        \item En ingress som är visuellt särskiljd ifrån själva inlägget (genom skiljd storlek, typsnitt, eller dyl.)
        \item En lång brödtext (det faktiska blogginlägget) bestående av flera stycken. Följande behöver förekomma i brödtexten.
          \begin{enumerate}
            \item Minst ett ord/begrepp i \emph{Fetstil}
            \item Minst ett ord/begrepp i \emph{Italics}
            \item Minst en bild
            \item Minst en \emph{Blockquote}
          \end{enumerate}
      \end{enumerate}

\subsection{ Mouse Over med CSS }
\paragraph{}
Vi ska nu träna på att göra mouseover-effekter med hjälp av css. Dels för länkar men även för lite andra grejer. När vi gör hovers i CSS så använder vi något som kallas för \emph{pseudo-elements}. Nedan är ett exempel:
%\lstset{language=CSS}
\begin{lstlisting}
#my-button:hover{
color: red;
}
\end{lstlisting}
Ovan kod css-regel gör alltså att vi applicerar stilregeln \emph{color: red; } på elementet \emph{\#my-button}, MEN endast när den är \emph{hover}:ed.

\subsubsection*{Uppgiften}
\begin{enumerate}
	\item Skapa en ny .html-sida
	\item Skapa en ny .css-fil
	\item Koppla in din .css-fil till din .html-sida
	\item Lägg till en \emph{<div>} i din .html-sida
	\item Lägg till en rubrik (\emph{<h1>}) i din div
	\item Lägg till ett stycke text (\emph{<p>}) i din div
	\item Lägg till en länk (\emph{<a>}) i din div
	\item Lägg till en bild (\emph{<img>}) i din div
	\item Skriv en css-regel som gör att bilden får en ram när man håller över den med musen.
	\item Skriv en css-regel som gör att bildens opacitet sänks när man håller över den med musen.
	\item Skriv en css-regel som gör att länken byter färg när man håller över den med musen.
	\item Skriv en css-regel som gör att länken blir fetstilt när man håller över den med musen.
	\item Skriv en css-regel som gör att div:en byter bakgrundsfärg när man håller över den med musen.
	\item Skriv en css-regel som gör att texten i div:en byter färg när man håller musen över div:en.
	\item Skriv en css-regel som gör att bilden i div:en ändrar storlek när man håller musen över div:en.
	\item Se till att du inte upprepar dig i dina css-regler.
	Med det menas att följande css...
	\begin{lstlisting}
	h1{
	color: red;
	}
	h1{
	font-weight:bold;
	}
	\end{lstlisting}
	... med fördel alltså kan och bör skrivas som följande:
	\begin{lstlisting}
	h1{
	color: red;
	font-weight: bold;
	}
	\end{lstlisting}
\end{enumerate}




  \newpage
  \section{ Centrera ett element }
    \paragraph{}
    I denna övning tränar vi på visuell centrering.

    \subsection*{Uppgiften}
      \begin{enumerate}
        \item
          Det som ska centreras är:
          \begin{enumerate}
            \item En rubrik
            \item En bild
            \item En brödtext
          \end{enumerate}
        \item Brödtexten ska vara "vänsterjusterad" i relation till rubriken och bilden men ändå vara i mitten av skärmen.
        \item Prova gärna att lösa denna uppgift genom alla tre följande metoder:
          \begin{enumerate}
            \item \texttt{ text-align:center; }
            \item \texttt{ position:absolute; }
            \item \texttt{ float:left; } och \texttt{ float:right }
          \end{enumerate}
      \end{enumerate}


  \newpage
  \section{ Spalter \& kolumner }
    \paragraph{}
    Denna uppgift går ut på att försöka dela upp en webbsida i spalter. Lägg ut ett antal (börja förslagsvis med två) \texttt{<div>}:ar och ge de genom CSS bredd, höjd och bakgrunds-färg. Detta så att vi kan skilja dem åt.

    \subsection*{Uppgiften}
      Om du har två \texttt{<div>}:ar är målet att få de bredvid varandra så att den ena \texttt{<div>}:en tar upp 50\% av skärmens bredd och den andra resterande utrymme. Tillsammans ska de täcka upp hela webbläsarens bredd.
      \begin{enumerate}
        \item Lägg 2 st \texttt{<div>}:ar bredvid varandra.
        \item Lägg 3 st \texttt{<div>}:ar bredvid varandra.
        \item Gör 2 st rader där den första raden innehåller 2 st kolumner och den andra raden 3 st kolumner. Med kolumner menas här \texttt{<div>}:ar bredvid varandra på samma sätt som ovan.
      \end{enumerate}
      \paragraph{}
      Precis som i uppgiften ovan kan vi lösa detta problem på många olika sätt. Prova gärna olika metoder!




 % IDEAS:
 % responsive


\end{document}