\documentclass{article}
\usepackage{amssymb}
\usepackage[T1]{fontenc}
\usepackage[utf8]{inputenc}
\usepackage{xcolor}
\usepackage{verbatim}
\usepackage{hyperref}
\hypersetup{%
  colorlinks=true,% hyperlinks will be black
  linkbordercolor=red,% hyperlink borders will be red
  pdfborderstyle={/S/U/W 1}% border style will be underline of width 1pt
}

% TODO: Example of another exercise
% 70's posters
% https://www.google.se/search?q=70s+poster&es_sm=91&source=lnms&tbm=isch&sa=X&ei=r3YDVNmyGOj4yQOJ0ICYDw&ved=0CAgQ_AUoAQ

\begin{document}

  \title{ VEKTORGRAFIK -- Mini-case }
  \author{ Grundläggande Multimedia | Uppsala Universitet }
  \date{}
  \maketitle

  \paragraph{}
  Detta dokument innehåller ett par olika case. Du behöver såklart inte färdigställa alla, så fokusera på de du finner viktiga/utmanande och försök göra de så bra du kan! Kom ihåg att du är fri att avvika ifrån instruktionerna om du är något på spåren!


  \newpage
  \section{ Minimalistisk app-ikon }
    \paragraph{}
    Att skapa
    \href{http://dribbble.com/shots/561492-Checklist?list=searches&tag=app_icon}{ikoner}
    för
    \href{http://dribbble.com/shots/1000228-Payment-System-App-Icon/attachments/117809}{smartphone-appar}
    har 
    \href{http://dribbble.com/shots/1040904--Go-Piano-App-Icon-Design}{blivit}
    något
    \href{http://dribbble.com/shots/1001782-New-Fork-App-Icon-Design/attachments/118149}{utav}
    en
    \href{http://dribbble.com/shots/114537-Wunderlist-Icon?list=searches&tag=app_icon}{tävling}
    för designers, där gränsen för
    \href{http://dribbble.com/shots/200993-Boxing-Glove-App-icon?list=searches&tag=app_icon}{detaljnivån}
    inte tycks ha
    \href{http://dribbble.com/shots/824210-waffle-iphone-icon?list=searches&tag=app_icon}{något stopp}.
    Även om designers har många sätt att show-case:a sina skicklighter är onekligen snygga app-ikoner en trend som blivit väl ansedd.

    \subsection*{ Uppgiften }
      Ditt
      \href{http://dribbble.com/shots/1013293-App-Branding?list=searches}{uppdrag}
      är
      \href{http://dribbble.com/shots/988508-Simple-iOS-Icons?list=searches}{alltså}
      att 
      \href{http://dribbble.com/shots/1043059-iDo-app-icon?list=searches}{ta fram}
      en, mycket mer
      \href{http://dribbble.com/shots/1139639-Heart-Icon?list=searches}{minimalistisk}
      än de ovan nämnda,
      \href{http://www.behance.net/gallery/Transparent-App-Icon/10898437}{design}
      för en
      \href{http://www.behance.net/gallery/Simple-Weather-App/10738379}{ny}
      eller befintlig
      \href{http://www.behance.net/gallery/Flat-icon/10071269}{app}.
      \begin{enumerate}
        \item Bestäm dig för en app (existerande eller hittepå)
        \item Ta fram en design i Illustrator
        \item Lägg ikonen mot en bakgrund och kanske med en text för att göra den mer estetiskt tilltalande för "kunden" (se ovan exempel-länkar så förstår du vad som åsyftas)
      \end{enumerate}



  \newpage
  \section{ Vektoriserad avatar }
    \paragraph{}
    Detta mini-projekt går ut på att skapa en
    \href{http://vector.tutsplus.com/tutorials/tools-tips/quick-tip-rapid-vector-portrait-process/}{vektoriserad avatar}
    av antingen dig själv eller någon annan. Helt enkelt en grov simplifikation av ett foto. Kolla in
    \href{http://chiragtheoo7.deviantart.com/art/vector-girl-face-173563059}{någon}
    av dessa
    \href{http://fad02fad.deviantart.com/art/Marilyn-Monroe-354215090}{exempel}
    för att skapa dig en bättre
    \href{http://oddhouse.deviantart.com/art/Portrait-70809410}{bild}
    av vad som åsyftas.
    \href{http://foolecho.deviantart.com/art/Be-188800517}{Tänk}
    på hur
    \href{http://j3concepts.deviantart.com/art/Self-Portrait-April-09-117753384}{olika}
    varje bild är, och fundera över vilka
    \href{http://farm2.static.flickr.com/1336/5163223561_ce4751bb83.jpg}{detaljer}
    som gör att de avger så, ifrån varandra, skilda intryck.

    \paragraph{}
      \href{http://vector.tutsplus.com/tutorials/illustration/tracing-a-vector-face-from-a-reference-photo/}{Här}
      har du två
      \href{http://vector.tutsplus.com/tutorials/illustration/how-to-create-a-simple-vector-avatar-from-a-stock-image/}{tutorials}
      som kan vara till hjälp.

    \subsection*{ Uppgiften }
      \begin{enumerate}
        \item Ta ett foto på dig själv med hjälp av Photo Booth (eller hämta en bild ifrån internet)
        \item "Vektorisera" dig
        \item Lägg en bakgrund under din bild
      \end{enumerate}

      


  \newpage
  \section{ Tunnelbane-karta }
    \paragraph{}
    Denna uppgift är av det enklare slaget. En stad (du väljer vilken) behöver printa sin tunnelbane-karta i ett extremt stort format. Således behöver de den i vektorformat, och det är där du kommer in. Om det är alldeles för många linjer som går i det tunnelbane-system du valt. Trace:a bara över några.

    \subsection*{ Uppgiften }
      \begin{enumerate}
        \item Leta rätt på en tunnelbane-karta som inte redan är i vektorformat
        \item Vektorisera den genom en manuell metod
      \end{enumerate}



  \newpage
  \section{ Vektorisera en device! }
    \paragraph{}
    Som du kanske kommer ihåg ifrån Photoshop-labben så finns det många, många lägen en designer behöver en vektoriserad representation av nya tekniska enheter. Om ett uppdrag t.ex. består av att designa ett interface. Alltså kan det hända att vi skulle behöva en
    \href{http://dribbble.com/shots/969976-Sketched-Minimus}{iPhone}, en
    \href{http://dribbble.com/shots/931519-Nexus-4-Vector-Mockup}{Nexus}, en
    \href{http://dribbble.com/shots/1143675-Flat-Macbook-Pro-Freebie}{Macbook}, en
    \href{http://dribbble.com/shots/971338-iMac-youMac-he-sheMac}{iMac}, en
    \href{http://www.pinterest.com/pin/536350636843223522}{Cinema Display}, eller någon annan av de populära enheterna på marknaden.

    \subsection*{ Uppgiften }
      \begin{enumerate}
        \item Välj en valfri existerande enhet
        \item Hitta ett fotografi av enheten
        \item Skapa en vektor-representation av enheten med fotot som utgångspunkt
      \end{enumerate}


     
  \newpage
  \section{ Fantasin sätter gränserna }
    \paragraph{}
    \href{http://www.behance.net/gallery/Prints-2012/3565675}{Många}
    jobbar
    \href{http://mattlyon.tumblr.com/}{helt}
    enkelt
    \href{http://www.pinterest.com/pin/433260426621559006/}{rätt}
    och
    \href{http://www.pinterest.com/source/c8six.com/}{slätt}
    med konst i Illustrator.

    \paragraph{}
    Här
    \href{http://www.pinterest.com/pin/183451384791435134/}{kommer}
    en
    \href{http://society6.com/product/Forest-Pastel-Rzc_Print}{hel}
    del
    \href{http://society6.com/product/Miami-vVL_Print}{ytterligare}
    exempel
    \href{http://society6.com/product/Balloon-ylb_Print}{på}
    ytterligare
    \href{http://society6.com/product/Boston-chI_Print}{tekniker},
    \href{http://society6.com/product/Colorful-Rectangles_Print}{resultat}
    och
    \href{http://society6.com/product/The-end-of-the-rainbow-POk_Print}{möjligheter}
    att
    \href{http://society6.com/product/Here-comes-the-sun-e7D_Print}{sälja}
    sina prints.

    \paragraph{}
    Eller vad
    \href{http://www.pinterest.com/pin/73324300155256876/}{sägs}
    om
    \href{http://www.pinterest.com/pin/223913412694866947/}{karaktärer}
    som skulle
    \href{http://www.pinterest.com/pin/219339444322006574/}{kunna}
    passa i
    \href{http://www.pinterest.com/pin/73324300156309389/}{en}
    eller
    \href{http://www.pinterest.com/pin/73324300156309421/}{en}
    annan
    \href{http://www.pinterest.com/pin/480970435175337969/}{barnbok}.

    \subsection*{ Uppgiften }
      ...är alltså precis vad du vill att den ska vara! :)













  \begin{comment}
    \section{ Valfri webb-kontroll }
      \paragraph{}
      Någonting annat som blivit något utav en design-tävling för grafiska designers, är UI-kontroller. D.v.s sliders, knappar, dropdowns, on-off-switches, checkboxar, o.s.v.

      \subsection*{ Uppgiften }
        \begin{enumerate}
          \item Steg
        \end{enumerate}

        \paragraph{Resurser}
          \begin{itemize}
            \item
              Länk
              \href{http://google.com/}{ någonstans }
          \end{itemize}



    \section{ Ikoner }
      \paragraph{}
      Lorem ipsum

      \subsection*{ Uppgiften }
        \begin{enumerate}
          \item Steg
        \end{enumerate}

        \paragraph{Resurser}
          \begin{itemize}
            \item
              Länk
              \href{http://google.com/}{ någonstans }
          \end{itemize}


    \section{ Metro: Kommunikation }
      \paragraph{}
      Lorem ipsum

      \subsection*{ Uppgiften }
        \begin{enumerate}
          \item Steg
        \end{enumerate}

        \paragraph{Resurser}
          \begin{itemize}
            \item
              Länk
              \href{http://google.com/}{ någonstans }
          \end{itemize}


    \section{ Arkitektur i perspektiv }
      \paragraph{}
      Lorem ipsum

      \subsection*{ Uppgiften }
        \begin{enumerate}
          \item Steg
        \end{enumerate}

        \paragraph{Resurser}
          \begin{itemize}
            \item
              Länk
              \href{http://google.com/}{ någonstans }
          \end{itemize}

  \end{comment}


\end{document}