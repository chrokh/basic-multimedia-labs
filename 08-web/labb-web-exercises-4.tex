\documentclass{article}
\usepackage{amssymb}
\usepackage[T1]{fontenc}
\usepackage[utf8]{inputenc}
\usepackage{xcolor}
\usepackage{color}
\usepackage{verbatim}
\usepackage{hyperref}
\usepackage{listings}
\lstdefinelanguage{CSS}{
    keywords={hover,color,background-image,margin,padding,font,weight,display,position,top,left,right,bottom,list,style,border,size,white,space,min,width, transform, transition, transition-property, transition-duration, transition-timing-function},
    sensitive=true,
    morecomment=[l]{//},
    morecomment=[s]{/*}{*/},
    morestring=[b]',
    morestring=[b]",        
    alsoletter={-},
    alsodigit={:}
}
\lstdefinelanguage{JavaScript}{}
\lstset{
  basicstyle=\color[rgb]{0.3,0.3,0.3}\ttfamily,
  keywordstyle=\color[rgb]{0,0.5,0.5},
  numberstyle=\color[rgb]{0.7,0.7,0.7},
  commentstyle=\color[rgb]{0.1,0.5,0.1},
  stringstyle=\color[rgb]{0.6,0.1,0.5},
  backgroundcolor=\color[rgb]{0.95,0.95,0.95},
  showstringspaces=false,
  numbers=left,
  breaklines,
  breakatwhitespace,
}

% No numbering
\setcounter{secnumdepth}{0}

% Counters for tasks & questions
\newcounter{taskcounter}
\setcounter{taskcounter}{0}
% Tasks
\newcommand{\task}[1]{
  \refstepcounter{taskcounter}
  \addcontentsline{toc}{subsection}{Uppgift \thetaskcounter{} #1}
  \vspace{4em}~
  \\\normalfont{\large{\bfseries{\hspace{0.5em}Uppgift \thetaskcounter \hspace{1em}#1}}}\\\\
}


\hypersetup{%
  colorlinks=true,% hyperlinks will be black
  linkbordercolor=red,% hyperlink borders will be red
  pdfborderstyle={/S/U/W 1}% border style will be underline of width 1pt
}

\begin{document}

  \title{ WEB IV -- Övningsuppgifter }
  \author{ Grundläggande Multimedia | Uppsala Universitet }
  \date{}
  \maketitle


  \section{ Introduktion }
    Även detta dokument innehåller små övningsuppgifter du kan använda för att träna dina multimediakunskaper i HTML-, CSS- och Javascript. Men nu börjar det bli svårare! Har du trubbel med att genomföra denna labb, gå då tillbaka och gör förra! Du kan jobba med samma webbsida i de flesta av de här uppgifterna, men om du vill så är det helt ok att skapa en ny webbsida per uppgift!
    \paragraph{}
    För att klara av den här labben behöver du vara inläst på \texttt{variabler} och \texttt{funktioner} samt \texttt{jQuery}-metoden \texttt{\$(document).ready()}. Du kan läsa på om dessa på \href{http://htmlhunden.se}{HTMLHunden.se}.

  \paragraph{}
  Eftersom vi kommer att jobba med webbsidor är det viktigt att du kommer ihåg några saker.
    \begin{enumerate}
      \item Se till att alla dina .html-dokument minst innehåller det minsta du behöver för att vara korrekt enligt HTML5-standarden (se nedan för ett exempel).
            \lstset{language=HTML}
            \begin{lstlisting}
  <!DOCTYPE html>
  <html>
    <head>
      <title>Your title here</title>
    </head>
    </body>
      <!-- Your content here -->
    </body>
  </html>
            \end{lstlisting}
      \item Se till att du \href{http://htmlhunden.se/#indentering}{indenterar korrekt}!
      \item Glöm inte att du alltid kan läsa igenom \href{http://htmlhunden.se}{HTMLHunden} om du känner dig förvirrad.
    \end{enumerate}





  \section{ Uppgifter }
  Nedan följer uppgifterna. Försök att hinna med allihopa!






  \task{ Välja element }
    Vi ska först träna på att använda css-selectors. Använd följande html-fil:
    \lstset{language=HTML}
    \begin{lstlisting}
<!DOCTYPE html>
<html>
  <head>
    <title>Your title here</title>
  </head>
  </body>
    <h1>Hello world</h1>
    <p>This is my page about superheroes.</p>
    <p>And here is a long text which you may replace with something longer like a set of lorem ipsum.</p>
    <div>
      <p>Some superheroes are..</p>
      <ul>
        <li><span>Bat</span>man</li>
        <li><span>Wonder</span>woman</li>
        <li><span>Iron</span> man</li>
        <li><span>Modesty</span> Blaise</li>
      </ul>
      <div>
        <p>And here's some more obscure superheroes..</p>
        <ul>
          <li><span>Powerpuff</span> girl 1</li>
          <li><span>No</span>manslander</li>
          <li><span>Wallpaper</span> woman</li>
          <li><span>Void</span> vendor</li>
        </ul>
      </div>
    </div>
  </body>
</html>
  \end{lstlisting}
  \paragraph{}
  Skapa nu en css-fil och koppla in den till dokumentet. Uppgiften går ut på att skriva följande CSS-regler, \textbf{utan} att förändra html-filen. Du får alltså \textbf{inte} lägga till klasser och ID:n i html-filen.
    \paragraph*{Uppgiften}
      \begin{enumerate}
        \item Gör rubriken (\texttt{<h1>}) understruken, och ge den en egen färg.
        \item Gör alla \texttt{<span>}-taggar som ligger i \texttt{<ul>}-listor fetstilta.
        \item Ge den \texttt{<div>} som innehåller den \textbf{andra} \texttt{<ul>}-listan en annan bakgrundsfärg.
        \item Ge den \texttt{<div>} som innehåller den \textbf{första} \texttt{<ul>}-listan en annan bakgrundsfärg.
        \item Gör så att första \texttt{<p>}-taggen efter rubriken har en lite större font-size och lite ljusare färg. Som en ingress i en tidning.
        \item Gör \textbf{vartannat} listelement i den \textbf{första} listan ljusgrått, och \textbf{vartannat} listelement i den \textbf{första} listan mörkgrått.
        \item Gör \textbf{vartannat} listelement i den \textbf{andra} listan rött, och \textbf{vartannat} listelement i den \textbf{andra} listan orange:t.
      \end{enumerate}





  \task{ setTimeout }
    JavaScript-metoden \texttt{setTimeout()} använder vi när vi vill fördröja exekveringen av någonting. Vi använder den som så:
    \lstset{language=[Sharp]C}
    \begin{lstlisting}
setTimeout(function(){
  // here is where we can do stuff
  alert("This message will be shown after 3000 ms");
  // like the alert box above
}, 3000);
    \end{lstlisting}

    \paragraph*{Uppgiften}
      \begin{enumerate}
        \item Skapa ett .js-dokument
        \item Koppla in .js-dokumentet till ditt .html-dokument
        \item Skicka ett alert-meddelande som i ovan exempel efter 3 sekunder.
      \end{enumerate}



  \task{ document fade out }
    Vi ska nu arbeta med jQuery.

    \paragraph*{Uppgiften}
      \begin{enumerate}
        \item Koppla in jQuery till ditt .html-dokument
        \item Skicka upp en alert-box med ett meddelande såsom "Hello from jQuery!" \textbf{när jquery \$(document).ready() är klar!}
      \end{enumerate}





  \task{ fadeOut, fadeIn }
    Vi ska nu försöka fade:a ut och in rubriken. Använd dig av jquery's metoder \texttt{fadeOut()} och \texttt{fadeIn()}.

    Anledningen till att vi innan tränade på css-selectors är att jQuery arbetar med just css-selectors. För att "välja" en html-nod genom jQuery så använder vi just css-selectors.

     \begin{lstlisting}
// selects all links
$('a');

// selects all <ul>-elements inside of <div>-elements
$('div ul');

// selects all elements with the ID "title"
$('#title');
    \end{lstlisting}

    \paragraph{}
    Vi använder \texttt{fadeIn()} och \texttt{fadeOut()} så här:

    \begin{lstlisting}
// fade:ar ut alla <li>-element under 4 sekunder
$('li').fadeOut(4000);

// fade:ar in alla element med klassen awesome
$(.awesome).fadeIn(2000);
    \end{lstlisting}

    \paragraph{}
    Om vi vill göra någonting när fade:en är klar behöver vi använda \texttt{callbacks}, t.ex. så här:
    \begin{lstlisting}
// fade:ar ut hela <body> och skickar sen upp en alert-box!
$('body').fadeOut(8000, function(){
  alert("OMG IT'S ALL GONE");
});
    \end{lstlisting}

    \paragraph*{Uppgiften}
      \begin{enumerate}
        \item När HTML:en är färdigladdad \texttt{\$document.ready()} så ska rubriken \texttt{<h1>} fade:as ut.
        \item När \texttt{<h1>}-taggen har fade:ats ut helt, ska den sedan fade:as in.
      \end{enumerate}




  \task{ En funktion }
    Nu ska vi träna på funktioner. Vi kan definiera funktioner som så:

    \begin{lstlisting}
var fadeOutBody = function(){
  $('body').fadeOut(8000); 
}
    \end{lstlisting}

    Och sedan anropa den som så:

    \begin{lstlisting}
fadeOutBody();
    \end{lstlisting}

    Vi kan även förstås definiera flera funktioner och sedan anropa funktioner ifrån funktioner. Föreställ dig att vi utgår ifrån följande kod:
\begin{lstlisting}
$('body').fadeOut(function(){
  alert("OMG FADE OUT!");
});
\end{lstlisting}
Då skulle vi kunna bryta upp de olika delarna i olika funktioner som separat utför arbete. Som så:
\begin{lstlisting}
var fadeOutBody = function(){
  $('body').fadeOut(8000, function(){
    alertMessage();
  });
}

var alertMessage = function(){
  alert("OMG FADE OUT!");
}

// Now when we call fadeOutBody();
// first body will fade out
// then we'll get the alert message
// just like before
fadeOutBody();
\end{lstlisting}

Eftersom den \texttt{anonyma funktionen} på rad 2 inte gjorde någonting annat i sin kropp (rad 3) än att anropa en annan funktion så hade vi lika gärna kunnat skicka in den funktionen snarare än den anonyma. Om detta låter förvirrande, strunta i det och fortsätt arbeta på samma sätt som ovan. Men vad vi hade kunnat skriva är alltså...

\begin{lstlisting}
var fadeOutBody = function(){
  $('body').fadeOut(8000, alertMessage);
}
\end{lstlisting}

Nästa överkurs-kommentar är att vi även hade kunnat skicka den andra funktionen som argument till den första. Som så:

\begin{lstlisting}
var fadeOutBody = function(theCallback){
  $('body').fadeOut(8000, theCallback);
}

var alertMessage = function(){
  alert("OMG FADE OUT!");
}

// Now when we call fadeOutBody();
// it will expect us to also pass it another
// function which will act as a callback
fadeOutBody(alertMessage);
\end{lstlisting}

    \paragraph*{Uppgiften}
      \begin{enumerate}
        \item Skriv en funktion du kallar \texttt{fadeOutBody()} som fade:ar ut \texttt{<body>}.
        \item Skriv en funktion du kallar \texttt{fadeInBody()} som fade:ar in \texttt{<body>}.
        \item Använd \texttt{setTimeout} för att anropa funktionen \texttt{fadeOutBody()} efter \texttt{1000 ms}.
        \item Använd \texttt{setTimeout} för att anropa funktionen \texttt{fadeInBody()} efter \texttt{3000 ms}.
      \end{enumerate}




  \task{ fadeOut, fadeIn, forever }
    Denna uppgift går egentligen ut på samma sak som ovan. Men poängen är nu att när body är utfade:ad så ska den fade:as in. Och vice versa. För evigt. Lite som att vi skapar en "blink"-effekt.

    Tips: Tänk på att metoder kan anropa andra metoder, och att du antagligen inte behöver \texttt{setTimeout()}.

    \paragraph*{Uppgiften}
      \begin{enumerate}
        \item När \texttt{<body>} har fade:ats ut, ska den fade:as in igen.
        \item När \texttt{<body>} har fade:ats in, ska den fade:as ut igen.
        \item Och så fortsätter det...
      \end{enumerate}

    Tips: Tänk på de två första punkterna ovan som funktioner.





  \task{ Append }
    Vi ska nu öva på att dynamiskt förändra content på våra sidor. När man klickar på en länk så ska sidan förändras.

    För att lägga till content någonstans på en sida så använder vi jQuery-funktionen \texttt{append()}. Som så...

    \begin{lstlisting}
// First we select an element,
// then we call the append-method on the element
// and give it the html string for header.
$('body').append('<h1>Hello world</h1>');
    \end{lstlisting}

    Notera att vi nu appendar content till \texttt{<body>}, men vi hade lika gärna kunnat appenda content till vilket annat element som helst. Prova kör ovan kod själv!

    För att kunna lösa uppgiften behöver vi även ta reda på hur jQuery's klick-metod fungerar. Den fungerar som så här:

    \begin{lstlisting}
// First we select an element,
// then we call the click-method
// then we give it a function
$('a').click(function(){

  // which sends an alert box
  alert("Hello world!");

  // then we return false so that the browser
  // will not "follow" the actual link
  //     (i.e. navigate away from the page)
  return false;
});
    \end{lstlisting}

    Notera alltså att vi även i detta exempel valt en viss selector, i detta fall \texttt{<a>}-elementet, alltså alla länkar. Men vi hade förstås kunnat binda denna klickmetod till ett specifikt \texttt{<a>}, eller något helt annat element.

    Tips: Glöm alltså inte att ha \lstinline{return false;} i din klick-metod. Prova utan så förstår du varför.

    \paragraph*{Uppgiften}
      \begin{enumerate}
        \item Lägg till två länkar \texttt{<a>}-element.
        \item När man klickar på den första länken så ska en text (\texttt{<p>})läggas till i slutet av sidan.
        \item När man klickar på den andra länken så ska en bild läggas till i slutet av sidan.
      \end{enumerate}





  \task{ Positionering }
    Nu ska vi göra en snabb övning i absolut positionering med css.
    \paragraph*{Uppgiften}
      \begin{enumerate}
        \item Lägg till två bilder i ditt .html-dokument
        \item Skriv en regel i ditt .css-dokument som med hjälp av css-regeln \texttt{position: fixed;} positionerar första bilden i mitten av sidan. Oavsett hur användaren scrollar.
        \item Skriv en regel i ditt .css-dokument som med hjälp av css-regeln \texttt{position: absolute;} positionerar andra bilden högst upp till höger på sidan, men som inte "ligger kvar" när användaren scrollar. Bilden ska alltså inte vara \texttt{fixed}.
      \end{enumerate}



  \task{ css()-metoden }
    Vi ska träna på att förändra sidans css genom jQuery. För att applicera css på ett element genom jQuery använder vi metoden \texttt{css()}. Som nedan:

    \begin{lstlisting}
// First we select an element,
// then we call the css-method
// passing it a css property as the 1st argument
// and a suitable value as the 2nd argument
$('p').css('border', '14px solid blue');
    \end{lstlisting}

    \paragraph*{Uppgiften}
      \begin{enumerate}
        \item Lägg till ytterligare en bild i ditt .html-dokument
        \item Lägg till ytterligare en länk i ditt .html-dokument
        \item Gör så att bilden med hjälp av \texttt{position: absolute;} positioneras längst ned till vänster, när man klickar på länken.
      \end{enumerate}





  \task{ Random }
  Skriv en metod som returnerar ett slumpmässigt tal mellan 75 och 200. Kalla metoden \texttt{randomNumber}.



  \task{ Appending cats }
   I denna uppgift ska vi göra så att sidan fylls med bilder på katter allteftersom användaren klickar på en länk.

    \paragraph*{Uppgiften}
      \begin{enumerate}
        \item Lägg till ytterligare en länk i ditt .html-dokument
        \item Skriv en javascript-funktion som du döper till \lstinline{addRandomCat}.
        \item Metoden ska append:a en bild på en katt genom (t.ex. ifrån \href{http://placekitten.com}{placekitten.com}) till \lstinline{<body>}
        \item Metoden som appendar bilder, ska använda sig av metoden \lstinline{randomNumber} du tidigare skrev, för att placera bilden på en slumpmässig position. Detta genom \lstinline{position: absolute;} som vi tidigare tränade på.
        \item När användaren klickar på länken ska metoden \lstinline{addRandomCat} köras.
        \item Med andra ord, när användaren klickar på länken ska alltså en bild på en katt läggas till i \lstinline{<body>} på en slumpmässigt utvald position.
      \end{enumerate}







  \task{ Dismissable box }
    Nu ska vi skapa en \texttt{<div>} som går att "stänga".

    \paragraph*{Uppgiften}
      \begin{enumerate}
        \item Lägg till en \texttt{<div>} till ditt dokument.
        \item Ge den klassen \texttt{.closable}.
        \item Lägg till en länk (\texttt{<a>}) i \texttt{<div>}:en.
        \item När användaren klickar på länken, ska boxen försvinna.
      \end{enumerate}

    TIPS: Kolla på jQuery-metoderna \texttt{.hide()} och/eller \texttt{.remove()}.




  \task{ Mouse-tracker }
    Denna uppgift går ut på att skriva ut muspekarens position på skärmen.

    \paragraph*{Uppgiften}
      \begin{enumerate}
        \item Skapa ytterligare ett \texttt{<div>}-element
        \item Lägg till två \texttt{<p>}- eller \texttt{<span>}-element i \texttt{<div>}:en.
        \item Ge de två elementen ID:na \texttt{mouse-x} resp. \texttt{mouse-y}.
        \item Använd jQuery för att hela tiden uppdatera dessa två värden med muspekarens nuvarande position.
      \end{enumerate}

    TIPS: Kolla på jQuery-metoden \texttt{.mousemove()}.












 

\end{document}